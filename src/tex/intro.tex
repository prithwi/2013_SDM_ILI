% mainfile: ../ltexpprt.tex
Surveillance reports published by health organizations is one of the
main resources for monitoring Influenza like illness (ILI) cases and for
years have been used by healthcare officials for public reactions.
However, these surveillance reports are published with a considerable
delay and hence, it takes a long time for health threats to become
visible to health officials. Therefore, in recent years, various
approaches have been proposed in the literature for monitoring and
prediction of influenza epidemics based on data mining and network
dynamics methods. These approaches work based on different data sources
such as search engine query statistics \cite{ref1}\cite{ref2}, social
media micro-blogs \cite{ref3}-\cite{ref7}, news articles \cite{ref8},
and weather data \cite{ref9}. A number of approaches use only one data
source (such as \cite{ref1} and \cite{ref4}) while some solutions work
based on multiple data sources (such as \cite{ref3} and \cite{ref10}). 

Different solutions have been proposed for different purposes. In a
number of solutions, the aim is to predict the actual number of ILI
cases in a specific society \cite{ref2}, \cite{ref1}. These methods are
mostly work based on regression algorithms. In the second category, the
main goal is to monitor the spatio-temporal distribution of ILI cases in
a way that can be used by healthcare officials for public reaction
\cite{ref3}, \cite{ref4}. The dynamic behavior of epidemics in a
population is also studied in a number of works in order to understand
the way that disease propagates through people and between cities
\cite{ref8}, \cite{ref11}.

One of the first prediction methods based on search engine queries was
reported in \cite{ref2}. In their approach, Ginsberg et al. build time
series based on Google search queries to track weekly ILI counts. In
addition to Google search trends, previously published values in CDC ILI
reports are used to learn a simple regression model that predicts weekly
influenza counts two weeks ahead of CDC ILI surveillance reports. In
\cite{ref1}, Yuan et al. usesearch query data of Baidu (a popular search
engine in China) for influenza outbreak prediction. In their approach
they use real-time search query results for specific keywords as well as
official statistics of influenza to build a regression model that can
predict influenza case count one or two weeks in advance. In addition to
different data sources, this method differs from \cite{ref2} in terms of
the used regression model. In \cite{ref2} a logit-linear model is used
while \cite{ref1} uses a simple linear regression model.

Relationship between climate data and influenza outbreak has been
studied in \cite{ref9}. Existence of seasonal cycles of influenza
epidemics in different climate regions have been known for a long time.
In \cite{ref9} Tamerius et al. studied this relationship by considering
climatic information from 78 globally distributedsites. Through logistic
regression they found out that in ¿cold-dry¿ and ¿humid-rainy¿
environments strong correlation exists between influenza epidemics and
weather conditions.

Online social media data plays an important role in ILI epidemic
prediction. In \cite{ref4}, Lee et al. proposed a real-time surveillance
system that uses Tweeter data for monitoring and prediction of influenza
and cancer in US. This method can be considered as a single data source
solution that works with real-time Tweeter data. In \cite{ref4}
different types of analysis are used including spatial, temporal, and
textual analysis. The purpose of spatial analysis is to determine the
distribution of disease in US while temporal analysis is used for
tracking the changes in number of tweets with specific keywords.
Furthermore, text mining is used for tracking the popularity of disease
types, symptoms, and treatments. Results of these analyses are
visualized separately and can be used by healthcare officials. An almost
similar approach was reported in \cite{ref7} that usesspatio-temporal
analysis of Tweeter data to monitor West Nile Virus.Diversity of words
in tweets has been studied in \cite{ref5} and \cite{ref6} . In
\cite{ref5}, Kanhabua and Nejdl use clustering methods to determine
important topics in Tweeter data. They constructed time series for
matched keywords and used Jaccard coefficient to determine temporal
diversity of tweets. They have noted that this temporal diversity may be
correlated with real-world ILI outbreaks. In \cite{ref6} authors studied
the dynamic of changes in tweets related to H1N1 virus.

There are a few works in the literature that address using multiple data
sources \cite{ref10}, \cite{ref3}. These works highly depend on social
media data as well as other public available resources. In \cite{ref10}
different data sources are compared based on different factors such as
reliability and timeliness. Multiple data sources such as news articles,
social media, and clinical reports are considered in this study. Authors
also proposed an integrated framework for a health surveillance system
that works with multiple data sources. In \cite{ref3}, Denecke et al.
proposed an event-based approach which can be used for early prediction
of ILI threats \cite{ref3}. In their method (M-Eco) they consider
multiple resources such as Tweeter, TV reports, online news articles,
and blogs. M-Eco is a bi-lingual system that works with information in
English and German and uses clustering to group documents in to
clusters. These clusters are then interpreted as signals and used for
event detection. The system is also based on supervised learning methods
that rely on signal definitions which have been entered by user.

Dynamics of influenza out breaks have been also studied for historically
influenza epidemics. In \cite{ref8} and \cite{ref12}, epidemics of 1918
have been studied to understand the behavior of an influenza epidemic in
a society. Historical news articles have been used by Ewing et al. in
\cite{ref8} to study the spread of influenza pandemic. Authors used data
mining and network analysis methods for this purpose. Topics are modeled
by using latent Dirichlet allocation. In \cite{ref12}, Eggo et al. use a
gravity model for propagation of disease between cities. They used
Bayesian Markov Chain Monte Carlo methods to estimates parameters of
their model. 

Dynamic behavior of influenza epidemic has been also studied based on
synthetic data sources to face with scalability and extensibility
problems \cite{ref11}. Network dynamic solutions are used in
\cite{ref11} to study the behavior of an epidemic issue in a society.
Spread of an infection through a network has been also studied as a
general problem in graph-mining \cite{ref13} \cite{ref14}. In
\cite{ref14} the main problem is to find the culprits in an epidemic
situation based on available data. In \cite{ref13} the main concern is
finding the epidemic threshold to distinguish between die-out regimes
and break out regimes.
