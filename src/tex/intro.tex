% mainfile: ../ltexpprt.tex
Surveillance reports published by health organizations is one of the
main resources for monitoring Influenza like illness (ILI) cases and for
years have been used by healthcare officials for public reactions.
However, these surveillance reports are published with a considerable
delay and hence, it takes a long time for health threats to become
visible to health officials. Therefore, in recent years, various
approaches have been proposed in the literature for monitoring and
prediction of influenza epidemics based on data mining and network
dynamics methods. A considerable number of such methods are based on collecting
social signals by mining diverse data sources such as 
search engine query statistics \cite{ref1, ref2}, social
media ``chatter''~\cite{ref3, ref4, ref5, ref6, ref7} and news articles \cite{ref8},


%and weather data \cite{ref9, Shaman_orig_humidity_link, Shaman_humidity_USA}.
%A number of approaches use only one data
%source (such as \cite{ref1} and \cite{ref4}) while some solutions work
%based on multiple data sources (such as \cite{ref3} and \cite{ref10}). 

%Different solutions have been proposed for different purposes. In a
%number of solutions, the aim is to predict the actual number of ILI
%cases in a specific society \cite{ref2}, \cite{ref1}. These methods are mostly 
%based on classical regression algorithms. At the same time, some research
%~\cite{ref3, ref4} has been directed at monitoring the spatio-temporal distribution of ILI cases in
%a way that can be used by healthcare officials for public reaction
%. The dynamic behavior of epidemics in a
%population is also studied in a number of works in order to understand
%the way that disease propagates through people and between cities
%\cite{ref8}, \cite{ref11}.
One of the pioneering work in this new-found usage of non-tradional data sources 
was presented by Ginsberg et al.~\cite{ref2} when the predicted week ILI case counts
by tracking the volume of search engine queries. This work inspired a number of similar 
researches such ~\cite{ref1}, where Yuan et al. used search query data from Baidu (a popular
search engine in Chine) to detect influenza outbreaks. Both these works were based on finding 
correlation between search query volumes of different keywords with official ILI estimates 
of corresponding regions. Apart from using different data sources, 
these two work also differed on the actual regression model used. While the firsst one was based on 
logit-linear model, the second one uses a simple linear regression model. 

%In their approach, Ginsberg et al. build time
%series based on Google search queries to track weekly ILI counts. In
%addition to Google search trends, previously published values in CDC ILI
%reports are used to learn a simple regression model that predicts weekly
%influenza counts two weeks ahead of CDC ILI surveillance reports. 
%In their approach
%they use real-time search query results for specific keywords as well as
%official statistics of influenza to build a regression model that can
%predict influenza case count one or two weeks in advance. In addition to
%different data sources, this method differs from \cite{ref2} in terms of
%the used regression model. In \cite{ref2} a logit-linear model is used
%while \cite{ref1} uses a simple linear regression model.
Even real-time ILI detection~\cite{ref4} systems have been proposed by comparing 
real-time Twitter data with disease statistics to monitor and predict ILI and cancer 
in USA. This work also focusses on the importance of tracking both spatial and temporal 
data. A similar work was presented by Sugamaran and Voss~\cite{ref7} to analyze 
spatio-temporal spread of the West Nile Virus using Twitter Data. 
Historical news articles have also been used by Ewing et al. in
\cite{ref8} to study the spread of influenza pandemic. Authors used data
mining and network analysis methods for this purpose. Topics are modeled
by using latent Dirichlet allocation. In \cite{ref12}, Eggo et al. use a
gravity model for propagation of disease between cities. They used
Bayesian Markov Chain Monte Carlo methods to estimates parameters of
their model. %This method can be considered as a single data source
%solution that works with real-time Tweeter data. 

Apart from these ``social sources'', there has also been considerable work on 
correlating physical indicators such as climate data with influenza outbreaks. 
The primary advantage of such data sources is the fact that the effects are much 
more causal and less noisy. 
Shaman et. al. ~\cite{ref9, Shaman_orig_humidity_link, Shaman_humidity_USA} 
explored this area in detail and found absolute humidity to be a good indicator of 
influezna outbreaks.

While these works have helped in showcasing the importance of the varied 
``surrogate'' sources that can be used to predict and/or monitor ILI outbreaks, 
only a few works have aimed at combining mutiple data sources 
~\cite{ref10, ref3}.  These works are mainly dependent on social
media data as well as other publicly available data sources such as news 
articles, social media and clinical reports. However, to the best of our knowledge 
ours is the first work that aims at investigating both ``social indicators'' as well 
as ``physical indicators'' to predict the ILI incidence. While, the social indicators
can serve to elucidate more direct response from the society about prevalent ILI 
conditions, physical indicators such as humidity provide direct observance about 
the underlying ILI levels. 

Also, from our analysis we have found that the official estimates as reported by
agencies~\cite{PAHO:2013} are often lagged by several weeks and even when they
are reported, the reports are modified upto several weeks before the estimates 
are finalized. A real-time prediction system which is based on historical data
needs to handle this kind of variations in the offical estimates to get an 
accurate picture of the ILI conditions. Finally, most of the analysis have been 
retrospective and without a proper experimental validation framework.

In this paper we propose a novel matrix-factorization based regression method with 
nearest neighbor embedding to predict the ILI case counts. Our contributions can be listed
as:
\begin{itemize}
  \item We investigate both ``social indicators'' and ''physical indicators'' to 
    understand the efficacy of each kind of sources towards ILI prediction.
  \item We present a systematic way of investigating each individual source for possible 
    correlation with ILI by describing a novel matrix factorization based regression method
    using neighborhood embedding to account for 
    non-linear relations between the surrogates and the official ILI estimates.
  \item We investigate te efficacy of combining the these different sources at model 
    level vs. data level towards final accuracy.
  \item We propose different ways of handling uncertaininties in the official 
    estimates and factor these uncertainities in our prediction models.
  \item Finally, we present a detailed and prospective analysis of our proposed methods
    by comparing predictions from a near-hoirzon real time prediction system to 
    official estiamtes.
\end{itemize}

The rest of the paper is organized as follows: we touch upon some related works in 
Section~\ref{sec:related}, followed by a formal problem definition in Section~\ref{sec:problem},
and description of our regression framework in Section~\ref{sec:methods}. In Section~\ref{sec:ensemble}, 
we investigate the question of how best to combine multiple sources and present our strategies to 
handle the variance in official estimates in Section~\ref{sec:moving}. We conclude with detailed 
experiments in Section~\ref{sec:experiments}, correspoding results in Section~\ref{sec:results} and present
our conclusions in Section~\ref{sec:conclusions}.
 %In \cite{ref10}
%different data sources are compared based on different factors such as
%reliability and timeliness. Multiple data sources such as news articles,
%social media, and clinical reports are considered in this study. Authors
%also proposed an integrated framework for a health surveillance system
%that works with multiple data sources. 

%In \cite{ref3}, Denecke et al.
%proposed an event-based approach which can be used for early prediction
%of ILI threats \cite{ref3}. In their method (M-Eco) they consider
%multiple resources such as Tweeter, TV reports, online news articles,
%and blogs. M-Eco is a bi-lingual system that works with information in
%English and German and uses clustering to group documents in to
%clusters. These clusters are then interpreted as signals and used for
%event detection. The system is also based on supervised learning methods
%that rely on signal definitions which have been entered by user.
