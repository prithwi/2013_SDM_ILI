% mainfile: ../ltexpprt.tex
Surveillance reports published by health organizations are one of the
primary resources for monitoring influenza like illness (ILI) cases and,
for years, have been the primary source of information used by healthcare
officials for policy decisions. However traditional surveillance
reports are published with a considerable delay and thus recent research
has focused on mining social signals from diverse data sources such as
search engine query volume~\cite{ref1, ref2}, and social
media chatter~\cite{ref3, ref4, ref5, ref6, ref7}.


%and weather data \cite{ref9, Shaman_orig_humidity_link, Shaman_humidity_USA}.
%A number of approaches use only one data
%source (such as \cite{ref1} and \cite{ref4}) while some solutions work
%based on multiple data sources (such as \cite{ref3} and \cite{ref10}). 

%Different solutions have been proposed for different purposes. In a
%number of solutions, the aim is to predict the actual number of ILI
%cases in a specific society \cite{ref2}, \cite{ref1}. These methods are mostly 
%based on classical regression algorithms. At the same time, some research
%~\cite{ref3, ref4} has been directed at monitoring the spatio-temporal distribution of ILI cases in
%a way that can be used by healthcare officials for public reaction
%. The dynamic behavior of epidemics in a
%population is also studied in a number of works in order to understand
%the way that disease propagates through people and between cities
%\cite{ref8}, \cite{ref11}.

One of the pioneering works in this 
space is the work of
Ginsberg et al.~\cite{ref2} where the authors 
predicted weekly ILI case counts
by tracking the volume of search engine queries. This work 
inspired significant follow-on work, e.g.,~\cite{ref1}, where Yuan 
et al. used search query data from Baidu (a popular search 
engine in China) to detect influenza outbreaks. 
More real-time ILI detection~\cite{ref4} systems have been proposed 
by modeling Twitter streams.
%with disease statistics to monitor and predict ILI and cancer 
%in USA. This work also focusses on the importance of tracking both spatial and temporal 
%data. A similar work was presented by Sugamaran and Voss~\cite{ref7} to analyze 
%spatio-temporal spread of the West Nile Virus using Twitter Data. 
%Historical news articles have also been used by Ewing et al. in
%\cite{ref8} to study the spread of influenza pandemic. Authors used data
%mining and network analysis methods for this purpose. Topics are modeled
%by using latent Dirichlet allocation. In \cite{ref12}, Eggo et al. use a
%gravity model for propagation of disease between cities. They used
%Bayesian Markov Chain Monte Carlo methods to estimates parameters of
%their model. %This method can be considered as a single data source
%solution that works with real-time Tweeter data. 

Apart from such social media sources,
there has also been considerable research on 
correlating physical indicators such as climate data with influenza outbreaks. 
The primary advantage of such data sources is that the
effects are much more causal and less noisy. 
Shaman et. al.~\cite{ref9, Shaman_orig_humidity_link, Shaman_humidity_USA} 
explored this area in detail and found absolute humidity 
to be a good indicator of influezna outbreaks.

While the above works have made important strides, there are two important areas that
have been relatively less studied. First, 
only a few works have focused on combining mutiple data sources~\cite{ref10, ref3}
to aid in forecasting. In particular, to the best of our knowledge there has been no work
that investigates the combination of social indicators and physical indicators to forecast
ILI incidence. Second, and more improtantly, official estimates as reported by health
organizations (e.g., WHO, PAHO)~\cite{PAHO:2013} are often lagged by several weeks and
even when reported are typically revised for several weeks before the case counts are
finalized. Real-time prediction systems must be designed to handle the forecasting of
such a `moving target'. Finally, most existing works have been retrospective and not set in
the context of a formal data mining validation framework. To overcome these deficiences, we
propose a novel approach to ILI case count forecasting. Our contributions are:
\begin{itemize}
  \item Our approach integrates both social indicators and physical indicators and thus
leverages the selective superiorities of both types of feature sets. We systematize such
integration using a novel matrix factorization-based regression approach
using neighborhood embedding, thus helping account for 
non-linear relationships between the surrogates and the official ILI estimates.
  \item We investigate the efficacy of combining diverse different sources at two
levels: data fusion level, and model level, and discuss the relative (de)merits.
  \item We propose different ways of handling uncertainties in the official 
    estimates and factor these uncertainities into our prediction models.
  \item Finally, we present a detailed and prospective analysis of our proposed methods
    by comparing predictions from a near-horizon real time prediction system to 
    official estimates of ILI case counts in 15 countries of Latin America.
\end{itemize}

%The rest of the paper is organized as follows: we touch upon some related works in 
%Section~\ref{sec:related}, followed by a formal problem definition in Section~\ref{sec:problem},
%and description of our regression framework in Section~\ref{sec:methods}. In Section~\ref{sec:ensemble}, 
%we investigate the question of how best to combine multiple sources and present our strategies to 
%handle the variance in official estimates in Section~\ref{sec:moving}. We conclude with detailed 
%experiments in Section~\ref{sec:experiments}, correspoding results in Section~\ref{sec:results} and present
%our conclusions in Section~\ref{sec:conclusions}.
 %In \cite{ref10}
%different data sources are compared based on different factors such as
%reliability and timeliness. Multiple data sources such as news articles,
%social media, and clinical reports are considered in this study. Authors
%also proposed an integrated framework for a health surveillance system
%that works with multiple data sources. 

%In \cite{ref3}, Denecke et al.
%proposed an event-based approach which can be used for early prediction
%of ILI threats \cite{ref3}. In their method (M-Eco) they consider
%multiple resources such as Tweeter, TV reports, online news articles,
%and blogs. M-Eco is a bi-lingual system that works with information in
%English and German and uses clustering to group documents in to
%clusters. These clusters are then interpreted as signals and used for
%event detection. The system is also based on supervised learning methods
%that rely on signal definitions which have been entered by user.
