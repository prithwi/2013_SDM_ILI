% mainfile: ../ltexpprt.tex

The keywords relating to ILI were
organized from a seed set of words and expanded using a combination of 
time-series correlation analysis and pseudo-query expansion.
The seed set of keywords (e.g., {\em gripe}) was constructed in Spanish, 
Portugese, and English using feedback from our 
in-house subject mattter expers.

\paragraph{Pseudo-query expansion.}
Using the seed set, we crawled
the top 20 web sites (according to Google Search) associated with each
word in this set. We also crawled some expert sites such as the official CDC
website and equivalent websites of the countries under consideration, detailing the
causes, symptoms and treatment for influenza.
Additionally we crawled a few hand-picked websites such as
\url{http://www.flufacts.com} and \url{http://health.yahoo.net/channel/flu\_treatments}.
We filtered the words from these sites using standard language
processing filtering techniques such as stopword removal and Porter
stemming. The filtered set of keywords were then ranked according to 
the absolute frequency of occurrence. The top 500 words for Spanish and
English were then selected. For example, words such as {\em enfermedad}
and {\em pandemia} were obtained from this step.

\paragraph{Time-series correlation analysis.}
Next we used Google Correlate (now a part of Google Trends) to identify keywords
most correlated with
the ILI case count time-series for each country.
Once again these words were found to be a mix of 
 both English and Spanish. As an added step in this process, we also
 compared time-shifted ILI counts: left-shifted  to capture the words searched leading up to 
 the actual flu infection and right-shifted to capture the words
commonly searched during the tail of the infection. 
This entire exercise provided us some interesting terms like {\em ginger} which has been used as
a natural herbal remedy in the eastern world. We also found popular flu medications
such as {\em Acemuk} and  {\em Oseltamivir}, which are also sold under the trade name of
{\em Tamiflu} as highly correlated search queries, especially particularly for
Argentina.

\paragraph{Final filtering.}
The set of terms obtained from query expansion and correlation analysis were then 
pruned by hand to obtain a vocabulary of 151 words. We then performed a final
correlation check and retained a final set of 114 words.

