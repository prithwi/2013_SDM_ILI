% mainfile: ../ltexpprt.tex

By using multiple methods like pseudo query expansion and correlation
analysis using Google Correlate we built a vocabulary of 151 words (both
in Spanish and English) that were found to be closely associated with
flu. To begin with, our subject matter experts provided us with a set of
words that were associated with flu. These included symptoms, medical
synonyms and effects. This set was used as the initial seed for pseudo
query expansion, wherein the top 20 web sites returned by Google search
for each of these words were crawled. Apart from this, the official CDC
website and equivalent websites in each country detailing the
causes,symptoms and treatment for influenza was also crawled.
Additionally some hand picked websites such as
\url{http://www.flufacts.com}
and \url{http://health.yahoo.net/channel/flu\_treatments} were also crawled.
This entire corpus was then normalized by removing stop words and using
porter stemming, post which the words were ranked according to the
number of occurrences. The top 500 words for each language was selected.
Words such as enfermedad (Spanish for disease) and pandemia were
obtained from this step.  Google Correlate (now a part of Google Trends)
allowed one to provide an arbitrary time series and it returned back a
list of words whose search volume was most correlated to the given time
series. For our vocabulary building exercise, the ILI case counts
published by PAHO on their official website
\url{http://ais.paho.org/phip/viz/ed\_flu.asp} was used to create a time
series. For each country, this time series was then fed into Google's
Correlate engine and the top 20 correlated words were obtained. These
words were a mix of both English and Spanish. As an added exercise, the
case count time series was shifted to the left by upto 2 weeks in order
to capture the words searched leading up to the actual flu infection.
Similarly, we shifted the time series to the right to capture the words
commonly searched during the tail of the infection. This entire exercise
provided us some interesting terms like ``ginger'' which has been used as
a natural herbal remedy in the eastern world and many popular flu
medications like ``Acemuk'' which is popular flu medication in South
America and  ``Oseltamivir'' which is sold under the trade name of
``Tamiflu'' were found to be highly correlated to the flu case count time
series particularly in Argentina.   The set of terms obtained from query
expansion and correlation analysis were then pruned by hand to obtain
the final vocabulary 151 words. These words were used as input for the
GST model. We then noticed 37 of these words had negligible to almost-no
weights that the model learnt during the training phase and hence were
omitted from the vocabulary, giving us a total of 114 keywords that are
predictive of flu outbreak in South America.
