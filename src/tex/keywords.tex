% mainfile: ../ltexpprt.tex

The keywords indicating Influenza like Illness (ILI) activites was
constructed from a set of seed words and expaned using a combination of 
techniques such as time-series correlation analysis and pseudo-query
expansion.
The set of starting keywords was constructed based on feedback from our 
in-house subject mattter expers. These included keywords such as 
``symptoms'', ``medical'' and ``effects''.
\paragraph{Pseudo-query Exapnsion}
Using this set as the initial seed for pseudo-query expansion we crawled
the top 20 web sites (according to Google Search) associated with each
word in this set. We also crawled some ``expert'' sites such as the official CDC
website and equivalent websites of the countries under consideration detailing the
causes, symptoms and treatment for influenza.
Additionally, we also crawled a few hand picked websites such as
\url{http://www.flufacts.com} and \url{http://health.yahoo.net/channel/flu\_treatments}.
We filtered the words from these sites using standard language
processing filtering techniques such as stopword removal and porter
stemming. The filtered set of keywords were then ranked according to 
the absolute frequency of occurence. The top 500 words for Spanish and
English were then selected. For example, words such as ``enfermedad'' 
and ``pandemia'' were obtained from this step.
\paragraph{Time-series Correlation analysis}
Next we used Google Correlate (now a part of Google Trends) to compare
the ILI case count time-series for each country and found the top 20
correlated words. Once again these words were found to be a mix of 
 both English and Spanish. As an added step in this process, we also
 compared time-shifter ILI counts: left-shited  to capture the words searched leading up to 
 the actual flu infection and right-shifted to capture the words
commonly searched during the tail of the infection. 
This entire exercise provided us some interesting terms like ``ginger'' which has been used as
a natural herbal remedy in the eastern world. We also found populaar flu medications
such as ``Acemuk''and  ``Oseltamivir'', which are also sold under the trade name of
``Tamiflu'' as highly correlated search queries - paritcularly for
Argentina.
\paragraph{Final Filtering}
The set of terms obtained from query expansion and correlation analysis were then 
pruned by hand to obtain a vocabular of 151 words. We performed a final
correlation check considering all the words and the ILI case counts and
removed all the words with correlation scores lower than a certain
threshold and were left with a final set of 114 keywords that are
predictive of flu outbreak in South America.
