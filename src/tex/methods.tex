% mainfile: ../ltexpprt.tex
\prithwish{Resolve conflicts for \\
1. ``time-point'' vs ``time point''\\
2. ``langle'' vs ``left(''}

In order to solve the problem formalized in Section~\ref{sec:problem}, we employ
non-linear regressions on the case count and the surrogate attributes. 
We describe two models:
\begin{itemize}
  \item Nearest Neighbor Based Regression.
  \item Matrix Factor Based Regression.
\end{itemize}
For both of the methods, we deine two paramters- $\beta$ and $\alpha$. 
$\alpha$ is the ``lookahead window length'' denoting the time-point from $T$ 
for which we want to predict the case count and $\beta$ 
is the ``lookback window length'' denoting the number of time-points to looback
in order to find the regression relation between the case count and the surrogate 
data.

Also, we define regression vectors $V_t$  and 
labels $L_t$ for all time points $t, \forall t = 1,\dots, T$
as given in equation~\ref{eq:regressionvector}.
\prithwish{Check this}
\begin{equation}
  \label{eq:regressionvector}
  \begin{array}{lcl}
    V_t & \equiv & \langle P_{t-\beta - \alpha}, \mathcal{X}_{t-\beta - \alpha}, P_{t + 1 -\beta-\alpha}, \mathcal{X}_{t + 1 - \beta-\alpha}, \dots, \\
        &        & P_{t-\alpha},\mathcal{X}_{t-\alpha} \rangle \\
    L_t & \equiv & P_{t - \alpha}\\
  \end{array}
\end{equation}
The the regression vector for predicting the case count at time point $T' (T +
\alpha > T' > T)$ is given by equation~\ref{eq:regressiontestvector}.
\begin{equation}
  \label{eq:regressiontestvector}
  \begin{array}{lcl}
    V_{T'} & \equiv & \langle P_{T'-\beta - \alpha}, \mathcal{X}_{T'-\beta - \alpha}, P_{t + 1 -\beta-\alpha}, \mathcal{X}_{t + 1 - \beta-\alpha}, \dots, \\
           &        & P_{T'-\alpha},\mathcal{X}_{T'-\alpha} \rangle \\
  \end{array}
\end{equation}

Under these definitions we describe the models as follows:

\subsection{\label{sec:model:nearestneighbor} Nearest Neighbor Based
Regression:}
For Nearest Neighbor Models, we define a training set $\Gamma_{NN}
= \lbrace V_t, L_t \rbrace$, where $V_t$ represents the regression attributes
and $L_t$ denote the corresponding labels. 

The predicted count $\widehat{P_{T'}}$ for the time point $T'$ is given as:
\begin{equation} \label{eq:nearestneighbor:pred}
  \begin{array}{lcl}
    \widehat{P_{T'}} & = & \frac{\sum\limits_{k=1}^{K} b_{k}L_{T^k}}{\sum\limits_{k=1}^{K} b_{k}}\\
  \end{array}
\end{equation}
where,$K$ indicates the maximum number of nearest neighbors considered, $T^k$
indicates the $k-th$ nearest neighbor to $V_{T'}$ in $\Gamma_{NN}$, and $b_k$
indicates the weight assigned to the $k^{th}$ nearest neighbor. Typically the 
inverse euclidean distances to $V_{T'}$ is chosen as the weights.

\subsection{\label{sec:model:matrixfactor} Matrix Factorization Based
Regression:}

