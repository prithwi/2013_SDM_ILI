% mainfile: ../ltexpprt.tex
%\prithwish{Resolve conflicts for \\
%1. ``time point'' vs ``time point''\\
%2. ``langle'' vs ``left(''}

We employ non-linear temporal regressions over the surrogate attributes to forecast the
case count using three broad models:
%\begin{itemize}
(a) Matrix Factorizaiton Based Regression (MF),
(b) Nearest Neighbor Based Regression (NN),  and
(c)  Matrix Factorization Regression using Nearest Neighbor embedding (MFN).
%\end{itemize}
For each of the methods, we define two parameters: $\beta$ and $\alpha$. 
$\alpha$ is the {\it lookahead window length}, denoting distance of the time point for prediction from $T$;
$\beta$ is the {\it lookback window length} denoting the number of time points to look back
in order to find the regression relation between the case count and the surrogate data.

We define regression vectors $V_t$  and 
labels $L_t$ for all time points $t, \forall t = 1,\dots, T$
as given in equation~\ref{eq:regressionvector}.
%\prithwish{Check this}
\vspace{-1em}
\begin{equation}
  \label{eq:regressionvector}
  \begin{array}{lcl}
    V_t & \equiv & \langle P_{t-\beta - \alpha}, \mathcal{X}_{t-\beta - \alpha}, P_{t + 1 -\beta-\alpha}, \mathcal{X}_{t + 1 - \beta-\alpha}, \dots, \\
        &        & P_{t-\alpha},\mathcal{X}_{t-\alpha} \rangle \\
    L_t & \equiv & P_{t}\\
  \end{array}
\end{equation}
The regression vector for predicting the case count at time point $T' (T +
\alpha > T' > T)$ is given by equation~\ref{eq:regressiontestvector}.
\begin{equation}
  \label{eq:regressiontestvector}
  \begin{array}{lcl}
    V_{T'} & \equiv & \langle P_{T'-\beta - \alpha}, \mathcal{X}_{T'-\beta - \alpha}, P_{t + 1 -\beta-\alpha}, \mathcal{X}_{t + 1 - \beta-\alpha}, \dots, \\
           &        & P_{T'-\alpha},\mathcal{X}_{T'-\alpha} \rangle \\
  \end{array}
\end{equation}
\noindent
Under these definitions we describe the models as follows:
\subsubsection{\label{sec:model:matrixfactor} Matrix Factorization Based Regression (MF):}
Matrix Factorization is a well accepted technique in
the recommender systems literature, to predict user 
preferences from incomplete user ratings/informations. Typically~\cite{canny2002factor}
a user-preference matrix is factored into an user-factor and
factor-preference matrix. However, such factorizations are incognizant of any 
temporal continuity. As such to enforce temporal continuity, to predict for the time point 
$T' (T +\alpha > T' > T)$ we use the regression vectors 
and labels as defined in equation~\ref{eq:regressionvector} to define a $m \times n$ {\it prediction
matrix} $\mathcal{M}$, as given in equation~\ref{eq:predictionmatrix}:
\vspace{-1em}
\begin{equation}
  \label{eq:predictionmatrix}
\mathcal{M} = \left[\begin{array}{ll}
              V_{\alpha + \beta + 1} & L_{\alpha+\beta + 1} \\
                              \vdots & \vdots \\
                               V_{T} & L_T \\
                               V_{T'} & L_{T'} \\ 
    \end{array}
  \right]
\end{equation}

The prediction matrix is factorized into a $f\times m$ factor-feature matrix $U$ and 
a $f\times n$ factor-prediction matrix as:
\[ \widehat{\mathcal{M}}_{i,j} = b_{u,i} + U^T_i \times F_j\]
Here, $b_{i,j}$ is the baseline estimate given by:
\vspace{-1em}
 \begin{equation}
   \label{eq:baseline}
   b_{i,j} = \bar{\mathcal{M}} + b_j 
 \end{equation}
 where $\bar{\mathcal{M}}$ represents the all-element average and $b_j$ represents 
 the column wise deviations from the average and is generally a free-parameter, i.e.,
it is fitted as part of the optimization problem.
$U$ and $F$ matrix are estimated by minimizing the error function:
\begin{equation}
  \begin{array}{ll}
  \label{eq:matrix:fit}
  b_*, F, U  & =  \textrm{argmin} (\sum \limits_{i=1}^{m-1} \left(\mathcal{M}_{i,n}  - \widehat{\mathcal{M}}_{i,n} \right)^2 \\ 
  & + \lambda_1\times(\sum \limits_{j=1}^{n}b_j^2 + \sum \limits_{i=1}^{m-1} || U_i||^2 + \sum \limits_{j=1}^{n} || F_j||^2))
\end{array}
\end{equation}
\noindent
where $\lambda_1$ is a regularization parameter. An important design criteria in
the error function of Eqn~\ref{eq:matrix:fit} is the fact that we only compute the error
between the predicted label values and the actual label values i.e., the $n^{th}$ column of the prediction 
matrix $\mathcal{M}$. The rationale behind this choice is the fact that unlike traditional recommender 
systems we are only concerned with the label column and can sacrifice reconstruction accuracies for other 
columns. 

The lookback window $\beta$, the factor size $f$ and the regularization parameter $\lambda_1$ 
are estimated using cross-validation
and the final prediction for time point $T'$ is given by:
\[\widehat{P}_{T'} = b_{m,n} + U_m^T \times F_{m,n} \]

\subsubsection{\label{sec:model:nearestneighbor} Nearest Neighbor Based Regression (NN):}
For nearest neighbor models, we define a training set $\Gamma_{NN}
= \lbrace V_t, L_t \rbrace$, where $V_t$ represents the regression attributes
and $L_t$ denote the corresponding labels.
Also, let us define the set 
$\mathcal{N}(i) = \lbrace k : \mbox{$V_k$ is one of the top  K nearest neighbors of $V_{i}$} \rbrace$ 
where $K$ indicates the maximum number of nearest neighbors considered.
The predicted count $\widehat{P}_{T'}$ for the time point $T'$ is given as:
\vspace{-1em}
\begin{equation} \label{eq:nearestneighbor:pred}
  \begin{array}{lcl}
    \widehat{P}_{T'} & = & \frac{\sum\limits_{k \in \mathcal{N}(T')} \theta_{k}L_{k,T - \alpha}}
    {\sum\limits_{k=1}^{K} \theta_{k}}\\
  \end{array}
    \vspace{-1em}
\end{equation}
%and $L_{k,T}$ indicates the label of the $k^{th}$ nearest neighbor to $V_{T'}$ in $\Gamma_{NN}$
%at time point $T$. 
\noindent
Here $\theta_k$ indicates the weight assigned to the $k^{th}$ nearest neighbor.
Typically the inverse Euclidean distances to $V_{T'}$ are chosen as the weights.

\subsubsection{\label{sec:model:nearestmatrix} Matrix Factorization Based
Regression using Nearest Neighbor Embedding (MFN):}
It has been shown in~\cite{koren2008factor} that matrix factorization using nearest neighbor constraints can
outperform classical matrix factorization approach as well as traditional nearest neighbor approaches towards
recommender systems. Here, we modify the method to suit the temporal nature of our problem in similar ways 
as described in section~\ref{sec:model:matrixfactor}. We again define a similar prediction matrix $\mathcal{M}$ 
(see equation~\ref{eq:predictionmatrix}). Following ~\cite{koren2008factor}, we define the 
matrix decomposition rule as 

\vspace{-1em}
\begin{equation}
  \label{eq:model:matrixfactornbr}
  \begin{array}{lcl}
    \widehat{\mathcal{M}}_{i,j} & = &  b_{i,j} + U_i^T\times F_j \\
                                &   & + F_j \times |\mathcal{N}(i)|^{-\frac{1}{2}}\sum_{k \in N(i)} (\mathcal{M}_{i,k} - b_{i,k}) x_k \\
  \end{array}
\end{equation}
\noindent
The key difference between equation~\ref{eq:model:matrixfactornbr} and the one proposed in 
~\cite{koren2008factor} is that we don't have any term for implicit feedback and, further,
only the top $K$ neighbors as found through Euclidean distance are used. 
The model is fitted using Eqn~\ref{eq:matrixnbr:fit} as given below:

\vspace{-1em}
\begin{equation}
  \label{eq:matrixnbr:fit}
  \begin{array}{ll}
    & b_*, F, U, x_*  = \textrm{argmin} (\sum \limits_{i=1}^{m-1} \left(\mathcal{M}_{i,n} - \widehat{\mathcal{M}}_{i,n}   \right)^2 \\
                    & + \lambda_2\times(\sum \limits_{j=1}^{n}b_j^2 + \sum \limits_{i=1}^{m-1} || U_i||^2 + \sum \limits_{j=1}^{n} || F_j||^2 + \sum_k ||x_k||^2))
  \end{array}
  \end{equation}

%\begin{equation}
%  \label{eq:matrixnbr:fit}
%  \begin{array}{ll}
%    b_*, F, U, x_*  = \textrm{argmin} &(\sum \limits_{i=1}^{m-1} \left(\mathcal{M}_{i,n} - \widehat{\mathcal{M}}_{i,n}   \right)^2 \\
%                    & + \lambda_2\times(\sum \limits_{j=1}^{n}b_j^2 + \sum \limits_{i=1}^{m-1} || U_i||^2 \\
%                    & + \sum \limits_{j=1}^{n} || F_j||^2 + \sum_k ||x_k||^2))
%  \end{array}
%  \end{equation}

