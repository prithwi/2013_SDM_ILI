% mainfile: ../ltexpprt.tex
Using the experimental setup we proceed to present our analysis about the efficacy 
of the proposed algorithms. In the process we also investiagte several possibilites 
and we next present our findings in accordance with such investigations. For all of 
these experimental results, we evaulated the algorithms over 6 months of predictions 
from Jan, 2013 to August, 2013. The final estimates of ILI case count was fixed according
to the estimates downloaded from PAHO on Oct 1, 2013 - a time frame with sufficient time
so that the values in our test set are stabilized. On this data set, we ran all the three different 
regressors as presented in Section~\ref{sec:methods} individually on all the surrogate sources (by setting
$\mathcal{X}$ to the corresponding surrogate time series)
as explained in the previous section (except OpenTable). The regressors were used to predict
2 weeks from the latest available PAHO ILI estimates for the corresponding calendar day. 
The findings of this endeavor is presented in Table.~\ref{tb:comparison_single}. We analyze 
some important observations from this table next

\begin{table*}[tb!]
\centering
\caption{\label{tb:comparison_single}  Comparing Prediction accuracy of regression models using individual sources.}
\vspace{1em}
\footnotesize
\begin{tabular}{|*{18}{c|}}
\hline
Model               & Sources       & AR & BO & CL & CR & CO & EC & GF & GT & HN & MX & NI & PA & PY & PE & SV & All\\
\hline \hline
\multirow{5}{*}{MF} & $\mathcal{W}$ &2.78&2.46&2.39&2.14&2.70&2.22&2.12&2.63&2.52&2.73&2.31&2.21&2.49&2.77&2.61&2.47\\ 
                    & $\mathcal{H}$ &2.81&2.31&2.22&1.92&2.43&2.04&2.11&2.57&2.33&2.48&2.39&2.15&2.18&2.47&2.33&2.32\\ 
                    & $\mathcal{T}$ &2.37&2.35&2.18&2.03&2.21&2.12&1.83&2.12&2.29&2.03&1.89&2.06&1.96&2.20&2.21&2.12\\ 
                    & $\mathcal{F}$ &2.34&2.11&2.29& N/A& N/A& N/A& N/A& N/A& N/A&2.71& N/A& N/A&2.31&2.24& N/A&2.33\\ 
                    & $\mathcal{S}$ &2.48&2.21&2.33&2.04&2.31&2.21&1.93&2.03&2.15&2.51&2.42&2.52&2.33&1.93&2.30&2.24 \\ 
\hline
\multirow{5}{*}{NN} & $\mathcal{W}$ &2.92&2.93&2.63&2.52&2.66&2.51&2.71&2.82&2.59&2.62&2.55&2.59&2.61&2.80&2.52&2.66\\ 
                    & $\mathcal{H}$ &2.73&3.10&2.42&2.27&2.83&2.64&2.43&2.25&2.71&2.31&2.61&2.35&2.43&2.39&2.52&2.53\\ 
                    & $\mathcal{T}$ &2.72&2.86&2.31&2.62&2.77&2.52&2.71&2.66&2.51&2.44&2.13&2.01&1.77&2.51&2.20&2.45\\ 
                    & $\mathcal{F}$ &2.11&2.21&2.33& N/A& N/A& N/A& N/A& N/A& N/A&2.19& N/A& N/A&2.41&2.32& N/A&2.26\\ 
                    & $\mathcal{S}$ &2.51&2.31&2.41&1.81&2.52&2.41&2.12&2.29&2.51&2.13&2.61&2.14&2.51&1.87&2.12&2.28 \\ 
\hline
\multirow{5}{*}{MFN}& $\mathcal{W}$ &2.99&3.01&2.88&2.53&2.78&2.81&2.77&2.83&2.61&2.70&2.56&2.66&2.82&2.79&2.51&2.75\\ 
                    & $\mathcal{H}$ &2.81&3.13&2.63&2.58&2.91&2.77&2.57&2.63&2.73&2.50&2.61&2.54&2.51&2.69&2.61&2.68\\ 
                    & $\mathcal{T}$ &2.74&3.03&2.51&2.64&2.83&2.51&2.81&2.71&2.60&2.48&2.13&2.55&2.19&2.57&2.31&2.57\\ 
                    & $\mathcal{F}$ &2.33&2.41&2.34& N/A& N/A& N/A& N/A& N/A& N/A&2.69& N/A& N/A&2.54&2.48& N/A&2.46\\ 
                    & $\mathcal{S}$ &2.61&2.44&2.55&2.22&2.61&2.52&2.71&2.31&2.62&2.48&2.61&2.31&2.53&2.23&2.13&2.46\\ 
\hline
\end{tabular}
\end{table*}


{\noindent \textbf{Google Flu Trends vs Google Search Trends: }}  Google Flu Trends
was one of the first and perhaps the most popular approach 
towards ILI outbreak detection using search query estimates. However, the Google
Flu Trends works on a closed-source keyword dictionary. We recreated the GFT setup 
by tracking search query volumes over our own custom ILI related keyword dictionary.
We evaluated the two different sources by using the same regression model on each 
data source and comparing the overall accuracy. As can be seen Table~\ref{tb:comparison_single},
for majority of the common countries (countries for which data from both Google Search Trends
and Google Flu Trends is present), regressors running on GST consistently 
outperforms those running on GFT. The only deviation could be seen in the case of Mexico, where the
accuracy from GFT trails the one from GST slightly. 
Thus we posit that the GST model so deviced is a a sufficiently close approximation to GFT, and 
with the added advantage of having access to raw level data and being available for more countries 
than GFT (among the 15 countries we consider, only 6 of them are present in the GFT database).


{\noindent \textbf{Optimal Regression model: }} From Table.~\ref{tb:comparison_single}, we can also
analyze the three different regressors proposed in Section.~\ref{sec:methods} with respect to overall accuracy.
With respect to each individual source, we can see that the Matrix Factorization with nearest 
neighbor embdedding performs the best in average over the countries, amonsgt the three.
Although, for some countties such as Panama while using only Google Search Trends, MFN works poorer to 
MF, the average accuracy over all countries for any given data source is best while using MFN.

\begin{table*}[tb!]
  \centering
  \caption{\label{tb:comparison_ensemble}Comparison of prediction accuracy while combining all data sources
  and using MFN regression.}
\vspace{1em}
  \begin{tabular}{|p{1.5cm}|*{16}{l|}}
\hline
Fusion Level& AR & BO & CL & CR & CO & EC & GF & GT & HN & MX & NI & PA & PY & PE & SV & All\\
\hline \hline
Model       &3.12&3.22&3.03&2.88&2.98&3.13&2.87&2.99&2.87&3.00&2.77&2.82&2.81&2.92&2.87&2.95\\ 
Data        &3.01&2.97&3.13&2.87&2.86&3.04&2.91&2.88&2.72&2.89&2.70&2.60&2.88&2.81&2.92&2.88\\ 
\hline
\end{tabular}
\end{table*}

{\noindent \textbf{Best way to combine the different data sources.: }} Another important 
question that needs to be investigated is to find out how best to combine these sources. 
As explained in Section.~\ref{sec:ensemble}, we proposed two different methods for fusing the
different sources: Model level vs Data level fusion. To analyse this, we ran the two different methods
using the Matrix Factorization based regression using nearest neighbor embedding which was found earlier 
found to be the best performing regression algorithm. The results are presented in Table~\ref{tb:comparison_ensemble}, 
which shows that the Model level fusion workds better overall. For 8 of the 15 countries, the model level fusion works
appreciably better than Data level fusion, while the reverse trend is seen for 4 other countries. 
This showcases the importance of considering both kinds of fusion as depending on the country of interest 
a particular type of fusion may be preferrable.

\begin{table*}[tb!]
  \centering
  \caption{\label{tb:moving} Comparison of prediction accuracy while using model level fusion 
  on MFN regressors and employing PAHO stabilization.}
\vspace{1em}
\begin{tabular}{|p{1.5cm}|*{16}{c|}}
\hline
Correction Method& AR & BO & CL & CR & CO & EC & GF & GT & HN & MX & NI & PA & PY & PE & SV & All\\
\hline \hline
None             &3.12&3.22&3.03&2.88&2.98&3.13&2.87&2.99&2.87&3.00&2.77&2.82&2.81&2.92&2.87&2.95\\ \hline
Weeks Ahead      &3.15&3.24&3.04&2.87&2.97&3.17&2.87&2.99&2.88&3.05&2.77&2.91&3.02&2.91&2.88&2.98\\ \hline 
Num samples      &3.20&3.24&3.03&2.88&2.96&3.12&2.87&3.01&2.89&3.12&2.78&2.92&3.04&2.91&2.87&2.99\\ \hline
Combined         &3.21&3.24&3.05&2.89&2.96&3.19&2.87&3.00&2.89&3.13&2.77&2.93&3.08&2.92&2.88&3.00\\ 
\hline
\end{tabular}
\end{table*}


{\noindent \textbf{Best way to combine the different data sources.: }} Thus finding the 
best in-average regressor and source fusion strategy, we investigate how we can use the 
official ILI case count estimate correction as presented in Section~\ref{sec:moving} to 
produce better quality predictions. As was shown, we can estimate the variability in the 
official ILI estimates either by looking at the difference in weeks between the day of upload 
and the week for which being the data is uploaded. At the same time, the number of samples used 
by the agencies to produce the estimates are also good indicators with lower number of samples
indicating variable estiamtes and high number of samples indicating more stable estimates. 
We test the correction procedures, by replacing the PAHO ILI case counts by their corrected values
and run the MFN over all data sources combined using model level fusion.
The results are reported in Table.~\ref{tb:moving}. As can be seen, corrected estimates 
using both the number of samples and 
the ``weeks ahead'' from the upload date, are generally better. It is to be noted that although
the correction is 
able to increase the overall accuracy only by a score 0.05 over all the countries, 
for some countries such as Mexico and Argentina (for which the data update is typically noisy) we get 
a substantial imporovement of scores. This indicates that the corrections may be selectively applied 
while predicting for certain countries. 


\begin{table*}[tb!]
  \centering
  \caption{\label{tb:Ablation} Discovering importance of sources in Model level fusion on MFN 
  regressors by ablating one source at a time.}
\vspace{1em}
\begin{tabular}{|*{17}{c|}}
\hline
Sources & AR & BO & CL & CR & CO & EC & GF & GT & HN & MX & NI & PA & PY & PE & SV & All\\
\hline 
\hline
All               & 3.21& 3.24& 3.05& 2.89& 2.96& 3.19& 2.87& 3.00& 2.89& 3.13& 2.77& 2.93& 3.08& 2.92& 2.88& 3.00\\
w/o $\mathcal{W}$ & 2.91& 2.99& 2.77& 2.71& 2.61& 2.59& 2.66& 2.69& 2.49& 2.78& 2.62& 2.87& 2.60& 2.43& 2.67& 2.69  \\
w/o $\mathcal{H}$ & 3.04& 2.85& 2.89& 2.56& 2.81& 2.77& 2.61& 2.75& 2.75& 2.82& 2.57& 2.75& 2.51& 2.87& 2.71& 2.75  \\
w/o $\mathcal{T}$ & 2.92& 3.14& 2.95& 2.61& 2.72& 2.81& 2.88& 2.79& 2.61& 2.93& 2.74& 2.63& 2.79& 2.74& 2.81& 2.80  \\
w/o $\mathcal{S}$ & 3.19& 3.11& 2.92& 2.64& 2.69& 2.70& 2.89& 2.88& 2.78& 3.07& 2.75& 2.91& 2.80& 2.71& 2.86& 2.86  \\
w/o $\mathcal{F}$ & 3.20& 3.12& 2.88& 2.89& 2.96& 3.19& 2.87& 3.00& 2.83& 3.02& 2.77& 2.93& 2.98& 2.88& 2.88& 2.96  \\
\hline
\end{tabular}
\end{table*}

{\noindent \textbf{Importance of data sources - Physical vs. Non-physical indicators: }} Finally we revisit the importance of different data sources
towards final prediction. From Table.~\ref{tb:comparison_single}, we can analyze the data sources
to find the ones which works the best when only individual models are created for each of them. 
In that analysis, the data source with the best accuracy happens to be the physical indicator source, 
weather data. However, while combining multiple sources the relations as identified via the individual models
may no longer hold true. As such we present a detailed ``ablation test'' where we removed one data source at a time
from the MFN and model level fusion framework and compared the final accruacy over the different countries. 
While removing the weather data, degrades the accuracy score the most, removing any source from the framework is found
to degrade the score to varying degrees. Thus we claim that it is important to consider both the physical 
and non-physical indicators to get a refined signal about the prevalent ILI incidence in the population.


\begin{itemize}
  %\item Table~\ref{tb:comparison_single} Comapring the three regression methods (one for each source and country). 
  %\item Discussion GST vs GFT.
  %\item Table~\ref{tb:comparison_ensemble} Which Ensembling method to use.
  %\item Table~\ref{tb:Ablation} Ablation tests
  %\item Table~\ref{tb:moving} Forecasting moving target (three different methods).
  \item Opentable data (more analysis... make this with only Mexico.. just a single plot may suffice). 
\end{itemize}


{\noindent \textbf{How good is restaurant reservation data to detect ILI: }} All the results so produced till now
didn't consider OpenTable reservations data. The reason is that this data is available only for Mexico among the 
countries being investigated. In general greater avaiability of tables may indicate higher ILI incidence rate in population.
We considered two different time winodws : Lunch and Dinner and used the number of open tables as the surrogate data.
We applied the three different regressors and also applied Model fusion over MFN regressors on this data and present 
our findings in Table~\ref{tb:opentable}. There we also compare the accuracy while using reservation 
data from only lunch and dinner as well as both time slots. As the results show, from this dataset we get the best 
result when we consider both lunch and dinner reservation data. Also, we find that including this data as part of the
ensemble decrease the accuracy by 0.01 over the one observed while using uncorrected ILI case count data. Thus its 
our opinion that although the reservation data does conceal some signals about prevlent ILI conditions, additional
information such as social-unrest events must be appended to this dataset to make this useful over the 
other described sources.

\begin{table}[tb!]
\centering
\caption{\label{tb:opentable}  ILI case count prediction accruacy for MX using OpenTable Data as a single source and
by combining it with all other sources using Model level fusion on uncorrected ILI case count data.}
\vspace{1em}
\begin{tabular}{|p{1.5cm}|*{2}{l|}p{2cm}|}
\hline
Method& Lunch & Dinner & Lunch and Dinner Time \\
\hline \hline
MF   & 1.92 & 2.23 & 2.31 \\
NN   & 1.99 & 1.83 & 2.11 \\
MFN  & 2.11 & 2.31 & 2.44 \\
Model Fusion & 2.96 & 2.87 & 2.99 \\
\hline
\end{tabular}
\end{table}


