% mainfile: ../ltexpprt.tex

In the last section, we described different regression strategies to correlate
a specific source with the ILI case count of a specfic country and predict
future ILI counts. However, we want to work with a multitude of different data
sources to get more accurate results. For this purpose, we broadly explored two
different ways to work with multiple data sources:

\begin{enumerate}
  \item Data level fusion, where we build a single regressor for all the different data
    sources.
  \item Model level fusion, where we build one regressor for each data source and 
    the combine the prediction from the models.
\end{enumerate}

In this section, we describe these fusion methods. Experiment results with each method is presented in Section~\ref{sec:results}.

\subsection{\label{sec:fusion:data} Data level fusion :}
In Data level fusion, we express the feature vector $\mathcal{X}$ as a tuple over all the different data 
sources and then proceed with either of the regression methods as outlined in Section~\ref{sec:methods}.
For example, while combining Twitter Data and Weather Data, the feature vector $\mathcal{X}$ is given 
by:
\[\mathcal{X}_t = \langle \mathcal{T}_t, \mathcal{W}_t \rangle
\]

\subsection{\label{sec:fusion:model} Model level fusion :}
Model level fusion is a more involved operation than the Data level fusion. 

The models are combined using matrix factorization regression with 
nearest neighbor embedding by comparing the
prediction estimates from each model with the actual estimate and the average
ILI case count for the month for the particular country.

Let us denote the average ILI case count for a particular calendar 
month $I$ for a given country by:
\begin{equation*}
  \mu_I = {1 \over {|\lbrace t \in I\rbrace|}} \sum_{t \in I}P_{t}
\end{equation*}
%where $|\mathcal{P}|$ is size of $\mathcal{P}$. 

Considering $C$ different sources and hence $C$ different models, 
let us denote the prediction for the $t^{th}$ time point 
from the $c^{th}$ model by ${}_c\widehat{P}_t$.

Under these definitions we can now proceed to describe the fusion 
model. Essentially, the model is similar to the one described in 
Section~\ref{sec:model:nearestmatrix}, with the difference being 
the way we construct the feature vectors. Similar to equation~\ref{eq:predictionmatrix},
we construct a prediction  $m'\times n'$ matrix for fusion given by${}_C\mathcal{M}$ where 
the $t^{th}$ row is represented by equation~\ref{eq:fusion:predictionmatrix}.

\begin{equation}
  \label{eq:fusion:predictionmatrix}
  {}_C\mathcal{M}_{t} = \left[\begin{array}{llll}
      {}_1\widehat{P}_{t}& \dots & {}_C\widehat{P}_t & P_t 
    \end{array}
  \right]
\end{equation}
Then similar to equation~\ref{eq:model:matrixfactornbr}
, we factor this matrix into latent factors, ${}_C U$, ${}_C F$, ${}_C b_*$ as 
given by equation\ref{eq:fusion:matrixfactornbr}
\begin{equation}
  \label{eq:fusion:matrixfactornbr}
  \begin{array}{l}
    {}_C \widehat{\mathcal{M}}_{i,j} =  \mu_i + {}_C b_{j} + {}_C U_i^T\times {}_C F_j \\
                                \qquad + {}_C F_j \times |{}_C \mathcal{N}(i)|^{-\frac{1}{2}}
    \sum_{k \in {}_C N(i)} ({}_C\mathcal{M}_{i,k} - \mu_i + {}_C b_{k}) {}_Cx_k \\
  \end{array}
\end{equation}
and the final prediction for the
$T$ data point is given by
\[\widehat{P}_T = {}_C \widehat{\mathcal{M}}_T,n'\].
The fitting function is given by equation~\ref{eq:fusion:matrixnbr:fit}:
\begin{equation}
  \label{eq:fusion:matrixnbr:fit}
  \begin{array}{l}
    {}_C b_*, {}_C F, {}_C U, {}_C x_*  = argmin (\sum \limits_{i=1}^{m'-1} \left({}_C \mathcal{M}_{i,n'} - {}_C \widehat{\mathcal{M}}_{i,n'}   \right)^2 \\
     \qquad + \lambda_3 (\sum \limits_{j=1}^{n'}{}_C b_j^2 + \sum \limits_{i=1}^{m'-1} ||{}_C U_i||^2 \\
     \qquad+ \sum \limits_{j=1}^{n'} ||{}_C F_j||^2 + \sum_k ||{}_C x_k||^2))
  \end{array}
\end{equation}

As before the free parameters are estimated through cross-validation.

