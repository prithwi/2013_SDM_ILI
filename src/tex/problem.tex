% mainfile: ../ltexpprt.tex

\begin{table*}[t!]
  \centering
  \begin{tabular}{|*{2}{l|}}
    \hline
    Abbreviation. & Description \\
    \hline \hline
    ${P}_t$         & Known ILI case count for time point $t$.\\
    $\hat{P}_t$     & Predicted ILI case count for time point $t$ \\
    $T$             & Max number of time points for which ILI case count is known.\\
    $\mathcal{X}_t$ & Surroagte data stream at time point $t$ .\\
    $T1$            & Max number of time points for which the surrogatedata streams are available.\\
    $\mathcal{F}_t$ & Google Flu Trends estimate at time point $t$ .\\
    $\mathcal{H}_t$ & Healthmap attributes at time point $t$ .\\
    $\mathcal{R}_t$ & Reservation data attributes at time point $t$ .\\
    $\mathcal{S}_t$ & Google Search Trends attributes at time point $t$.\\
    $\mathcal{T}_t$ & Twitter attributes at time point $t$ .\\
    $\mathcal{W}_t$ & Weather attributes at time point $t$ .\\
    $\alpha$        & Lookahead window length.\\
    $\beta$         & Lookback window length.\\
    $K$             & Maximum number of Nearest Neighbors.\\
    \hline
  \end{tabular}
  \caption{\label{tb:notations} Explanattions of notations used in the paper.
  \prithwish{May take this off later and put in the supplemental
section.}}
\end{table*}

Let
\[\mathcal{P} = \langle {P}_1, {P}_2, \dots,{P}_T \rangle\]
denote the known total weekly ILI case count for the country under
consideration, where ${P}_t$  denotes the case count for
time point $t$ and $T$ denotes the time point till which the
ILI case count is known.
%\prithwish{Will have to decide later whether we want to specify PAHO
  %here $\rightarrow$ \textnormal{``as reported by the Pan-American Health Organization
%(PAHO)~\cite{PAHO:2013}''}}\\
Corresponding to the ILI case count data, let us denote the available surrogate informations
for the same country by 
\[\mathcal{X} = \langle \mathcal{X}_1, \mathcal{X}_2, \dots, \mathcal{X}_{T1}\rangle,\]
where $T1$ is the time point till which the surrogate
information is available and $\mathcal{X}_{t}$ denotes the surrogate attributes for time
point $t$. 
The problem we desire to solve is to find a predictive model ($f$) for the 
case count data, as presented formally in Eqn~\ref{eq:problem}.
\begin{equation}
  \label{eq:problem}
  f: \mathcal{P}_t = f\left(\mathcal{P}, \mathcal{X}\right)
\end{equation}


