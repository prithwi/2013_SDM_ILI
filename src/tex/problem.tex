% mainfile: ../ltexpprt.tex

\begin{table*}[t!]
  \centering
  \begin{tabular}{|*{2}{l|}}
    \hline
    Abbreviation. & Description \\
    \hline \hline
    ${P}_t$ & Known ILI case count for time point $t$.\\
        $T$ & Max number of time-points for which ILI case count is known.\\
    $\mathcal{X}_t$ & Surroagte data stream at time point $t$ .\\
    $\mathcal{W}_t$ & Weather attributes at time point $t$ .\\
    $\mathcal{T}_t$ & Twitter attributes at time point $t$ .\\
    $\mathcal{H}_t$ & Healthmap attributes at time point $t$ .\\
    $\mathcal{R}_t$ & Reservation data attributes at time point $t$ .\\
    $\mathcal{F}_t$ & Google Search Trends attributes at time point $t$.\\
    ${S}_t$ & Google Flu Trends estimate at time point $t$ .\\
    \hline
  \end{tabular}
  \caption{\label{tb:notations} Notations used in the paper.
  \prithwish{May take this off later and put in the supplemental
section.}}
\end{table*}

In this section, we formally describe the problem.
Let $\mathcal{P} = \langle {P}_1, {P}_2, \dots,
{P}_T \rangle$ denote the known ILI case count for the country under
consideration , where ${P}_t$  denotes the case count for
the time point $t$.
Also let $T$ denote the time-point till which ILI case count is known.
\prithwish{Will have to decide later whether we want to specify PAHO
  here $\rightarrow$ \textnormal{``as reported by the Pan-American Health Organization
(PAHO)~\cite{PAHO:2013}''}}
Corresponding to the ILI case count data, let us denote the available surrogate informations
for the same country by $\mathcal{X} = \langle \mathcal{X}_1, \mathcal{X}_2, 
\dots, \mathcal{X}_{T1}\rangle$, where $T1$ is the time-point till which the surrogate
information is known and $\mathcal{X}_{t}$ denotes the surrogate attributes for time
point $t$. Then we want to find a predictive model ($f$)  for the case count dataas given in  
equation~\ref{eq:problem}.
\begin{equation}
  \label{eq:problem}
  f: \mathcal{P}_t = f\left(\mathcal{P}, \mathcal{X}\right)
\end{equation}


