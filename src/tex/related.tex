% mainfile: ../ltexpprt.tex
%In \cite{ref4}
%different types of analysis are used including spatial, temporal, and
%textual analysis. The purpose of spatial analysis is to determine the
%distribution of disease in US while temporal analysis is used for
%tracking the changes in number of tweets with specific keywords.
%Furthermore, text mining is used for tracking the popularity of disease
%types, symptoms, and treatments. Results of these analyses are
%visualized separately and can be used by healthcare officials. An almost
%similar approach was reported in \cite{ref7} that uses spatio-temporal
%analysis of Twitter data to monitor West Nile Virus. 

\begin{figure}[t] \centering
  \captionsetup{font=scriptsize}
    \includegraphics[width=0.95\columnwidth]{fig/ili_data_pipeline.pdf}
    \caption{\label{fig:ili_data_pipeline} Our ILI data pipeline, depicting 
six different data sources used in this paper to forecast ILI case counts.}
\vspace{-2em}
 \end{figure}

Related work naturally falls into the categories of social media
analytics, physical indicators, and event dynamics modeling.  These are
next described as follows:
 
\textbf{Social media analytics:}
Most relevant works using social media analytics focuses on Twitter,
specifically  by tracking
a dictionary of ILI-related keywords in the data stream.
Such investigations have often focused on the importance of diversity in keyword 
lists, e.g.,~\cite{ref5, ref6}. In~\cite{ref5}, 
Kanhabua and Nejdl used clustering methods to determine
important topics in Twitter data, constructed time series for matched keywords,
and used Jaccards coefficient to characterize the temporal
diversity of tweets. They noted, that such temporal diversity may be
correlated with real-world ILI outbreaks. In~\cite{ref6}
the authors studied the dynamics between the change in circulated tweets
and the H1N1 virus. Inspired by these works, we curated a custom ILI
related keyword dictionary which is described in details in Section~\ref{sec:keyword}.

\textbf{Physical indicators for detecting ILI incidence levels:} 
%One of the key indicators of ILI levels in a country is the prevalent 
%climactic patterns and the deviation from ``normal'' levels of these 
%indicators. For example 
Tamerius et al.~\cite{ref9} investigated the existence of seasonal 
cycles of influenza epidemics in different climate regions. For the said
work, they considered climatic information from 78 globally distributed sites. 
Using logistic regression  they found that, strong correlations exist between 
influenza epidemics and weather conditions, especially when conditions
are cold-dry or humid-rainy. Similarly, exciting results were reported
by Shaman et. al. in ~\cite{Shaman_orig_humidity_link, Shaman_humidity_USA}
where they discovered absolute humidity to be a key indicator of flu. To uncover 
these relationships they used non-linear regressors such as Kalman filters,
and this was a key inspiration for us in finding a uniform model for the
varied data sources as explained in Section~\ref{sec:methods}.

%\textbf{Non-linear regression methods:}
%There has been significant research in recommender systems to predict unknown user 
%ratings by combining techniques such as
%matrix factorization and nearest neighbor modeling. Although
%working essentially on categorical data, these methods are fine examples of non-linear
%regression methods which have been found to be robust as well as scalable (see~\cite{koren2008factor}).
%While matrix factorization have been long used (see ~\cite{canny2002factor}) to 
%connect the independent and dependent variables through latent factors, 
%recently Koren et al.~\cite{koren2008factor} presented detailed comparisons
%of nearest neighbors with matrix factorization methods and provided frameworks to 
%integrate the two approaches towards a unified non-linear predictor.
%
\textbf{Event dynamics modeling:}
Denecke et al.~\cite{ref3}
proposed an event-based approach for early prediction
of ILI threats \cite{ref3}. Their method (M-Eco) considers
multiple resources such as Twitter, TV reports, online news articles,
and blogs and uses clustering to identify signals for event detection.
%Also, there exists related research in modeling the progression of influenza epidemics using
%dynamical systems. 
Network dynamic solutions have also been used~\cite{ref11} 
to study the behavior of an epidemic in a society. 
%Spread of an infection through a network 
%has been also studied as a general problem in graph mining, e.g., see~\cite{ref14}.
 
