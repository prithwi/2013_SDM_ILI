% mainfile: ../ltexpprt.tex
Modern epidemiological forecasts of
common illnesses such as the flu rely on both traditional surveillance
sources as well as digital surveillance data such as social network
activity and search queries. 
However, most published
studies have been retrospective. For a real-time prediction system,
we posit that one of the key challenges is to effectively handle the uncertainty
associated with reports of flu activity.
Such reports are in general lagged by several weeks and typically revised
for several weeks after they are first reported. In this paper, 
we present a detailed
prospective analysis of the generation of robust quantitative predictions 
of temporal trends of flu activity using several {\it surrogate} data 
sources for 15 Latin American countries.
%In this paper we present a detailed
%prospective analysis about how we can generate robust quantitative predictions  
%about flu spread using several other ``surrogate data'' for 15 Latin
%American countries. 
We present our findings about the limitations and
advantages of correcting the uncertainty associated with official 
flu estimates. We 
also compare the prediction accuracy between model-level fusion 
of different surrogate data sources
against data-level fusion. Finally, we present a novel matrix factorization 
approach using neighborhood embedding to predict flu case counts. 
Comparing our proposed ensemble method against several baseline
methods helps us demarcate the importance of different data
sources for the countries under consideration.
