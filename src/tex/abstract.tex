% mainfile: ../ltexpprt.tex
In recent times, there has been a lot of interest in forecasting spread characterisitcs of
Influeza like Illness. Increasingly, traditional flu surveillance data
is being supplemeneted with ``digital surveillance'' data such as, social network
activities and search queries, to predict the onset/peak of the flu
season with better accuracy and lead time. However, most of the
studies have been conducted in the forrm of retrospective analysis. For a realtime prediction
system we found that one of the key challenges is to effectively handle the 
uncertainity associated the flu-surveillance estimates about the population. 
Such estimates are in general lagged by serveral weeks and often revised
for several weeks after they are first reported. In this paper we present a detailed 
'prospective' analysis about how we can generate robust quantitative predictions 
about flu spread using several other ``surrogate data'' for 15 Latin
American countries. In the process, we present our findings about the limits and
advantages of correcting offical flu-estimates based on update date and 
number of samples considered to construct the estimate. We also compare the 
prediction accuracy between model level fusion of different surrogate data sources
and data level fusion of the same fusion. Finally, we also present a
novel matrix factoization based method using neigborhood embdeding to
predict flu characteristics. We compare the proposed ensemble method
with several baseline methods and present our findings about the
importance of different data sources for the different countries under
consideration.
