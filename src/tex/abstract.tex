% mainfile: ../ltexpprt.tex
Recent times has witnessed a surge of interest in forecasting epidemiological 
characteristics of common diseases such as influenza like illnesses (the flu).
Traditional surveillance data is nowadays supplemented with digital surveillance data
such as social network activity and search queries to predict the onset/peak of
the flu season with greater accuracy and lead time. However, most published
studies have been retrospective analyses. For a real-time prediction system,
we posit that one of the key challenges is to effectively handle the uncertainty
associated with flu surveillance estimates about the population.
Such estimates are in general lagged by several weeks and often revised
for several weeks after they are first reported. In this paper we present a detailed
prospective analysis about how we can generate robust quantitative predictions  
about flu spread using several other ``surrogate data'' for 15 Latin
American countries. In the process, we present our findings about the limits and
advantages of correcting official flu-estimates based on update date and
the number of samples required to construct the new estimates. We also compare the
prediction accuracy between model level fusion of different surrogate data sources
against data level fusion. Finally, we present a novel matrix factorization based 
method using neighborhood embedding to predict flu characteristics. 
We compare the proposed ensemble method against several baseline methods, and 
present our findings about the importance of different data 
sources for different countries under consideration