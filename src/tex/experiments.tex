% mainfile: ../ltexpprt.tex
We used a number of different data sources from widely different domains
such as physical indicators like weather characterisitcs to social
networking/user queries. For a number of sources such as Twitter data,
the most important part is to properly identify the The keywords
associated with flu so that we can get a proper estimate of the flu
realted surrogate information. In section~\ref{sec:keyword}, we describe
our keyword collection mechanism as follows.
\subsection{\label{sec:keyword} Keywords Extraction}
% mainfile: ../ltexpprt.tex

The keywords relating to ILI were
organized from a seed set of words and expanded using a combination of 
time-series correlation analysis and pseudo-query expansion.
The seed set of keywords (e.g., {\em gripe}) was constructed in Spanish, 
Portugese, and English using feedback from our 
in-house subject mattter expers.

\paragraph{Pseudo-query expansion.}
Using the seed set, we crawled
the top 20 web sites (according to Google Search) associated with each
word in this set. We also crawled some expert sites such as the official CDC
website and equivalent websites of the countries under consideration, detailing the
causes, symptoms and treatment for influenza.
Additionally we crawled a few hand-picked websites such as
\url{http://www.flufacts.com} and \url{http://health.yahoo.net/channel/flu\_treatments}.
We filtered the words from these sites using standard language
processing filtering techniques such as stopword removal and Porter
stemming. The filtered set of keywords were then ranked according to 
the absolute frequency of occurrence. The top 500 words for Spanish and
English were then selected. For example, words such as {\em enfermedad}
and {\em pandemia} were obtained from this step.

\paragraph{Time-series correlation analysis.}
Next we used Google Correlate (now a part of Google Trends) to identify keywords
most correlated with
the ILI case count time-series for each country.
Once again these words were found to be a mix of 
 both English and Spanish. As an added step in this process, we also
 compared time-shifted ILI counts: left-shifted  to capture the words searched leading up to 
 the actual flu infection and right-shifted to capture the words
commonly searched during the tail of the infection. 
This entire exercise provided us some interesting terms like {\em ginger} which has been used as
a natural herbal remedy in the eastern world. We also found popular flu medications
such as {\em Acemuk} and  {\em Oseltamivir}, which are also sold under the trade name of
{\em Tamiflu} as highly correlated search queries, especially particularly for
Argentina.

\paragraph{Final filtering.}
The set of terms obtained from query expansion and correlation analysis were then 
pruned by hand to obtain a vocabulary of 151 words. We then performed a final
correlation check and retained a final set of 114 words.



\subsection{Data sources}
We next desribe the different data sources as follows: 
\subsubsection{Twitter} 
\subsubsection{Healthmap}
\subsubsection{Weather Data}
\subsubsection{Google Flu Trends}
\subsubsection{Google Search Trends}
\subsubsection{Open table}
We monitored table availability using OpenTable; an online restaurant
table reservation site for cities in the USA and Mexico. Our analysis
can be summarized as follows. First, using the OpenTable site, we
searched for the number of restaurants with available tables for two
persons at lunch and dinner. Since different regions and individuals
have different eating habits, we defined the lunch period between
12–3:30pm and dinner between 6–10:30pm. We searched for available tables
every hour and half past the hour for every day of the week. Next, we
investigated any occurrences of social unrest and natural disasters,
which might have affected the trend in the time series. Lastly, using
moving averages, cross-correlations and regression models, we elucidated
and compared the time-trend in the data of table availabilities to data
collected for various disease outbreaks. In the USA, we examined table
availability for restaurants in Boston, Atlanta, Baltimore and Miami.
For Mexico, we studied table availabilities in Cancun, Mexico City,
Puebla, Monterrey, and Guadalajara.
--We started collecting data on table availabilities starting September
4th 2012.
--The number of restaurants available varies over time since restaurants
are added and removed from the site. At the time of this writing, there
were over 28,000 restaurants in the database.
--Since we monitored ten regions at twenty search times, this resulted
in 200 distinct time-series curves.
--Size of data: The data files are not too big. When I analyzed it, each
was about 1kb which implies about 200kb in total.


