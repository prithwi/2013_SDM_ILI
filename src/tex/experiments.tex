% mainfile: ../ltexpprt.tex
We used a number of different data sources from widely different domains
such as physical indicators like weather characterisitcs to social
networking/user queries. For a number of sources such as Twitter data,
the most important part is to properly identify the The keywords
associated with flu so that we can get a proper estimate of the flu
realted surrogate information. In section~\ref{sec:keyword}, we describe
our keyword collection mechanism as follows.
\subsection{\label{sec:keyword} Keywords Extraction}
% mainfile: ../ltexpprt.tex

By using multiple methods like pseudo query expansion and correlation analysis using Google Correlate we built a vocabulary of 151 words (both in Spanish and English) that were found to be closely associated with flu. To begin with, our subject matter experts provided us with a set of words that were associated with the flu. These included symptoms, medical synonyms and effects. This set was used as the initial seed for pseudo query expansion, wherein the top 20 web sites returned by Google search for each of these words were crawled. Apart from this, the official CDC website and equivalent websites in each country detailing the causes,symptoms and treatment for influenza was also crawled. Additionally some hand picked websites such as http://www.flufacts.com and http://health.yahoo.net/channel/flu_treatments were also crawled. This entire corpus was then normalized by removing stop words and using porter stemming, post which the words were ranked according to the number of occurrences. The top 500 words for each language was selected. 
Google Correlate is/was a part of Google Trends which allowed users to query the most correlated words in terms of their search volume. Correlate also had a nifty tool wherein one could provide an arbitrary time series and get back a list of words whose search volume was most correlated to the given time series. For our vocabulary building exercise, the ILI case counts published by PAHO on their official website http://ais.paho.org/phip/viz/ed_flu.asp was used to create a time series. For each country, this time series was then fed into Google's Correlate engine and the top 20 correlated words were obtained. These words were a mix of both English and Spanish. As an added exercise, the case count time series was shifted to the left by upto 2 weeks in order to capture the words searched leading up to actual flu infection. Similarly, we shifted the time series to the right to capture the words commonly searched during the tail of the infection. This entire exercise provided us some interesting terms like "ginger" which has been used as a natural herbal remedy in the eastern world. Also, many medications like oseltamivir and tamiflu were found to be highly correlated.   
The set of terms obtained from query expansion and correlation analysis were then pruned by hand to obtain the final vocabulary used by the model. 


\subsection{Data sources}
We next desribe the different data sources as follows: 
\subsubsection{Twitter} 
\subsubsection{Healthmap}
\subsubsection{Weather Data}
\subsubsection{Google Flu Trends}
\subsubsection{Google Search Trends}
\subsubsection{Open table}
We monitored table availability using OpenTable; an online restaurant
table reservation site for cities in the USA and Mexico. Our analysis
can be summarized as follows. First, using the OpenTable site, we
searched for the number of restaurants with available tables for two
persons at lunch and dinner. Since different regions and individuals
have different eating habits, we defined the lunch period between
12–3:30pm and dinner between 6–10:30pm. We searched for available tables
every hour and half past the hour for every day of the week. Next, we
investigated any occurrences of social unrest and natural disasters,
which might have affected the trend in the time series. Lastly, using
moving averages, cross-correlations and regression models, we elucidated
and compared the time-trend in the data of table availabilities to data
collected for various disease outbreaks. In the USA, we examined table
availability for restaurants in Boston, Atlanta, Baltimore and Miami.
For Mexico, we studied table availabilities in Cancun, Mexico City,
Puebla, Monterrey, and Guadalajara.
--We started collecting data on table availabilities starting September
4th 2012.
--The number of restaurants available varies over time since restaurants
are added and removed from the site. At the time of this writing, there
were over 28,000 restaurants in the database.
--Since we monitored ten regions at twenty search times, this resulted
in 200 distinct time-series curves.
--Size of data: The data files are not too big. When I analyzed it, each
was about 1kb which implies about 200kb in total.


